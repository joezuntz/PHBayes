\documentclass[useAMS,usenatbib,a4paper]{mn2e}
\title[Bayesian Planet Hunting]{A Bayesian approach to PlanetHunters.org data\thanks{This publication has been made possible by the participation of the volunteers of the PlanetHunters project. Their contributions are individually acknowledged at http://www.planethunters.org/XXXXXXXXXX}}
\author[Lintott et al.]{
\parbox[t]{16cm}{Chris J. Lintott$^{1,2}$\thanks{Email: cjl@astro.ox.ac.uk}, Joe Zuntz$^{2,3}$ and the Planet Hunters team\\}\\ 
$^{1}$Adler Planetarium, 1300 S. Lakeshore Drive, Chicago, IL 60616, USA\\
$^{2}$Oxford Astrophysics, Denys Wilkinson Building, Keble Road, Oxford, OX1 3RH, UK\\
$^{3}$Department of Physics, University College London, Gower Street, London, WC1E 6BT, UK\\
}

\usepackage[dvips]{graphicx}
\usepackage{setspace}

\begin{document}

\date{April 2011}
\pagerange{\pageref{firstpage}--\pageref{lastpage}} \pubyear{2011}

\maketitle

\label{firstpage}

\begin{abstract}
Abstract abstract abstract

\end{abstract}

\begin{keywords}
Some keywords
\end{keywords}

\section{Introduction}

Finding planets is really important. Planet Hunters is a citizen science project from the Zooniverse which provides an independent analysis of data from NASA's \emph{Kepler} satellite. Volunteers inspect light curves, looking for transits, but need the awesome power of Bayesian statistics to answer the question of the number of planets. 

\section{Synthetics}

If not part of Meg's paper, write about the synthetics here

\section{Bayesian analysis}
%Mostly notes from first conversation in canteen after .Astronomy

Our goal is to estimate from Planet Hunters classifications the number of planets in the Kepler field, and their mass distribution ie

\begin{equation}
p\left(n\left(m_i\right)f_t\left(m_i\right)|data\right)
\end{equation}

where $m_i$ is the mass of a particular planet, $i$, $\mathrm{n\left(m_i\right)}$ is the fraction of stars with a planet of mass $m_i$. \footnote{What are we going to do with multiple planets? Or doesn't it matter?}

Insert justification for Bayesian magic.

\begin{eqnarray}
\lefteqn{p\left(\mathrm{data}|n\left(m_i\right)f_t\left(m_i\right)\right)=} \nonumber \\ & p\left(\mathrm{data_{synth}}|n\left(m_i\right)f_t\left(m_i\right)\right) \nonumber  \\ & \times p\left(\mathrm{data_{notsynth}}|n\left(m_i\right)f_t\left(m_i\right)\right)
\end{eqnarray}

But $p\left(\mathrm{data_{synth}}|n\left(m_i\right)f_t\left(m_i\right)\right)$ is the probability of getting $\mathrm{data_synth}$ transits per star, which follows a binomial distribution with $\mathrm{p\left(transit\right)}=f_t$ for given mass. 

And

\begin{eqnarray}
\lefteqn{p\left(\mathrm{data_{notsynth}}|n\left(m_i\right)f_t\left(m_i\right)\right)=} \nonumber \\ & \prod_{\mathrm{stars}}\left(p\left(f'_t\left(m_i\right)|f_tn\left(m_i\right)\right)\right) \nonumber \\ & + \left(p\left(\left(1-f'_t\left(m_i\right)\right)|f_tn\left(m_i\right)\right)\right)
\end{eqnarray}

where $f'_t$ is the fraction actually clicked\footnote{Presumably this isn't mass dependent because we don't know. So what happened to the mass dependence?}. These are just binomial functions too, and for a given model we can now calculate everything and fit to our results. 

So we need to marginalize over $f_t$ which means we need a model with $n$ points. The only priors we have are that for sufficiently large $m_i$ then $f_t=1$, and for sufficiently small $m_i$ $f_t=0$. We need some thought as to what `sufficiently' means here. The function is monotonic, and increasing. Joe is going to do some Gaussian-ness.

We also need a prior on the mass distribution. Chris is going to look at pre-launch Kepler papers to see if there's a prediction we can use as a basis. 



\begin{thebibliography}{99}


\end{thebibliography}

\label{lastpage}

\end{document}